%%%%%%%%%%%%%%%%%%%%%%%%%%%%%%%%%%%%%%%%%%%%%%%%%%%%%%%%%%%%%%%%%%%%%%%%%%%%%%%%%%%%%
% Machine Learning in Prediction
%
% 1. Linear Models
% 2. Polynomial Models
% 3. Neural Networks
% 4. Linear parameter-varying models
% 5. Adaptive nature using ISIS algorithm
% 
%%%%%%%%%%%%%%%%%%%%%%%%%%%%%%%%%%%%%%%%%%%%%%%%%%%%%%%%%%%%%%%%%%%%%%%%%%%%%%%%%%%

Cheap data storage and escalation of computational power allowed the world to enter a new age: the age of \textit{big data}. With vast amounts of data, previously in-viable and data hungry machine learning algorithms are now implementable.  The technology sector was the first group to be able to exploit this arcane technology to create tremendous value in applications ranging from targeted advertisements to self driving cars. The value was so great that the current top four companies in America by market capitalization are all technology companies (Microsoft, Apple, Amazon, and Facebook) as of 2019. As the technology sector's successes grow, other industries begin to catch a glimpse of the potential value creation in their own respective industries and initiate their own digital revolution. The ripples of success from the technology industry ultimately resulted in waves of capital investments into machine learning (ML) and artificial intelligence (AI) from all industries.

ML solutions promise to be cheaper, more accurate, and have online learning abilities compared to traditional methods.  Additionally, the solutions are promised to be easier to implement and will take less time to design; feed it data and it will learn, as they claim.  With this mentality, machine learning engineers and data scientists from technology companies attempted to conquer other industries, one industry being process control and chemical engineering. Unfortunately, their crusade fell short and their successes were few due to their lack of engineering knowledge and inability to identify large value gains.  Typically, projects in technology companies deal with very unambiguous information such as identifying location of objects or predicting the enjoyments of an individual based on previous articles they have read.

However, process control data are often times very ambiguous with data characteristics unique to the industry. Some characteristics include time delayed data, critically important yet unavailable data (data obtained through lab testing once every few hours), and highly noisy and/or unreliable sensor data.

Talk about how the methods are applied to process industry, compared to other places like cat classification.  Things such as time delay, type of model (steady state or dynamic), etc. that are unique to process control.

- Largest customer is probably soft sensors, training, and planning for management.  A field now named the industrial internet of things (IIoT). The industrial internet of things (IIoT) is the use of smart sensors and actuators to enhance manufacturing and industrial processes.

The objective of this section is to convey ideas for implementing machine learning solutions catered towards the \textbf{process control} industry.  In this chapter

The machine learning methods were then applied to the prediction and optimization of an industrial pipeline\footnote{This project was supported in part by Mitacs through the Mitacs Accelerate program.}. This chapter only contains the theory, application, and highlights. The detailed industrial project report can be found in Appendix A.

\section{Linear Models}

\section{Polynomial Models}

\section{Deep Learning Approaches}

\section{Linear Parameter-Varying Models}

\section{Adaptive Modelling: Importance Sampling Incremental Learning (ISIS)}
