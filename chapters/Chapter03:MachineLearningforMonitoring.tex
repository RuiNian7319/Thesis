%%%%%%%%%%%%%%%%%%%%%%%%%%%%%%%%%%%%%%%%%%%%%%%%%%%%%%%%%%%%%%%%%%%%%%%%%%%%%%%%%%%%%
% Machine Learning in Monitoring
%
% 1. Anomaly detection
%  - Adaptive using classifier updates
% 2. Alarm management
%  - Adaptive using RL
% 
%%%%%%%%%%%%%%%%%%%%%%%%%%%%%%%%%%%%%%%%%%%%%%%%%%%%%%%%%%%%%%%%%%%%%%%%%%%%%%%%%%%

ML prediction applications are effective complements to existing infrastructure in the process industry through soft sensors, state estimation, and forecasting. However, they are limited in applications regarding safety and risk management.  In the process industry, safety is upheld as the greatest value; investing in a successful safety system is just good business.  

\begin{quote}
    "Safety is a \textbf{value}, not a priority.  Priorities change, but company values never do." \\
    --- Rex Tillerson, ex-CEO of ExxonMobil
\end{quote}

Traditionally, process safety investments are frowned upon by management due to its high costs and \textit{invisible} returns. Indeed, a perfect safety and risk management system results in \textit{no change} in day-to-day activities because all the incidents are proactively mitigated.  However, if safety takes a back seat, the occurrence of the next incident is not a matter of if, its a matter of \textit{when}. Therefore, safety must be proactively (not reactively) managed to safeguard people, the environment, company assets and production capabilities. Here, ML can be leveraged to proactively monitor process systems and create an additional layer of safety. In this chapter, ML algorithms will be applied to predict equipment failures, process abnormalities, process variability and also perform alarm management. Through these applications, ML will be used to create multi-variate alarm systems that explore multi-variable interaction effects and gives fewer false alarms. Additionally, a new alarm management system that specifically tackles alarm flood scenarios will be introduced.  The objectives of this system are twofold: 1) Reduce sheer number of alarms during a flooding scenario; 2) identify the most important alarms so operators can prioritize safety critical alarms.

This chapter is organized as follows: Section 1 introduces data pre-processing methods for anomaly detection/prediction applications where the data is heavily imbalanced.  Section 2 introduces the anomaly detection and prediction algorithms and section 3 concludes this chapter with an introduction to a novel approach for alarm management.

The main contributions of this chapter are the data pre-processing methods used to prepare data sets for anomaly detection/prediction.  Additionally, it was shown that using synthetic data was able to enhance accuracy.  Lastly, a novel alarm management approach based on reinforcement learning was introduced to filter nuisance alarms and sort alarms based on their priority.  




Heavily unbalanced data:
- SMOTE, ADASYN, Breaking time down into smaller segments and increasing sampling rate

\section{Data Pre-processing for Monitoring}

\subsection{Data set preparation for anomaly detection}
- take only the previous few data points..

\subsection{Data set preparation for anomaly prediction}

\subsection{Synthetic data generation}


\section{Anomaly detection and prediction}
The model structure for logistic regression is given as:
\begin{equation}
    \hat{y} = \frac{1}{1 + e^{-(W_1^Tx + W_2^Tu + b)}}
    \label{eq:02LogS}
\end{equation}

- Measure process variability?

\subsection{Deep Learning Approaches}
\subsection{DL - Classification}

\subsection{Explanability}
- Explanability through weights, through importance sampling

\section{Alarm Management}

\subsection{Alarm Prioritization}

\subsection{Alarm Reduction}
